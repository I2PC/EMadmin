\documentclass[12pt,a4paper]{article}
%
% AUTHOR: Roberto Marabini__This is it
%
\usepackage{helvet} 
\renewcommand{\familydefault}{\sfdefault}
\usepackage{a4wide}
\usepackage{ucs}
\usepackage[utf8x]{inputenc}
\usepackage[english]{babel}
\usepackage{graphicx}
\usepackage[absolute]{textpos}
\usepackage{tabularx} 
\usepackage{tabulary}                 
\usepackage{fancyhdr}
\usepackage[table]{xcolor}
\usepackage{color}
\usepackage{pgf}
\newif\iflong

% %%%%%%%%%%%%%%%%%%%%%%%%%%%%%%%%%%%%%%%%%%%%%%%%%%%%%%%%%
% HEADER
\begin{document}
\raisebox{-.25\height}
{\color{blue}{\LARGE Electron Microscopy Facility at CNB/CIB-CSIC}}\vspace{1cm}\\
% FORM HEADER
\begin{tabularx}{1.0\textwidth}{|X|l|l|}
\hline
 \textbf{MICROSCOPE USAGE REPORT}
         & \textbf{Date:}
         & \VAR{acquisitionDate}\\\hline
{\textbf{User:} \VAR{acquisitionUserName}}
         & \textbf{ID:}
         &  \VAR{acquisitionId} \\\hline
{\textbf{Project:} \VAR{acquisitionProjName}}
         & \textbf{Sample:}
         & \VAR{acquisitionSample} \\\hline
{\textbf{Backup Dir:} \VAR{acquisitionBackupPath}}
         & \textbf{Shift (days):}
         & \VAR{acquisitionShiftLength}\\\hline
\end{tabularx}

\section*{\\Data Acquisition Report}
\def\arraystretch{1.25}  % cell padding  1 is the default, 
                         % change whatever you need
% MICROSCOPE
\begin{tabulary}{\linewidth}{L L}
  \hline\cellcolor{blue!25}Microscope & \cellcolor{blue!25}\\\hline
  Microscope ID & \VAR{microscopeName}\\\hline
  Model & \VAR{microscopeModel}\\\hline
  Detector & \VAR{microscopeDetector}\\\hline
  Physical Pixel Size ($\mu$) & \VAR{acquisitiondetectorPixelSize}\\\hline
  Voltage ($kV$)& \VAR{acquisitionVoltage}\\\hline
  Spheric Aberration ($mm$)& \VAR{microscopeCs}\\\hline
\end{tabulary}
%
  \hspace{1cm}
%
% DATA ACQUISITON
\begin{tabulary}{\linewidth}{L L}
  \hline
  \cellcolor{blue!25}Acquisition Params & \cellcolor{blue!25} \\\hline
  Nominal Magnification & \VAR{acquisitionNominalMagnification} \\\hline
  Sampling Rate ($\AA/px$)& \VAR{acquisitionSamplingRate}\\\hline
  Spot Size & \VAR{acquisitionSpotSize} \\\hline
  Illuminated Area ($\mu$)& \VAR{aquisitionIlluminatedArea}\\\hline
   & \\
   & \\
\end{tabulary}\\\\
% DOSE
   \newline
%
\begin{tabulary}{\linewidth}{L L}
  \\\hline
  \cellcolor{blue!25}Dose \& Fractions & \cellcolor{blue!25} \\\hline
  Dose per Fraction ($e/\AA^2$) & 
      \VAR{acquisitionDosePerFraction} \\\hline
  Total Exposure Time ($s$)& 
      \VAR{acquisitionTotalExptime} \\\hline
  Number Fractions & 
      \VAR{acquisitionNumFrames} \\\hline
  Frames per Fraction & 
      \VAR{acquisitionFramesPerFrac}\\\hline
  \cellcolor{blue!25}Apertures & \cellcolor{blue!25} \\\hline
  C2  & \VAR{acquisitionC2} \\\hline
  O1  & \VAR{acquisitionO1} \\\hline
  PhP & \VAR{acquisitionPhP} \\\hline
\end{tabulary}
%
  \hspace{1cm}
%
% EPU PARAMETERS
\begin{tabulary}{\linewidth}{L L}
  \\\hline
  \cellcolor{blue!25}EPU parameters & \cellcolor{blue!25} \\\hline
  Defocus Range ($\mu$)& \VAR{acquisitionNominalDefocusRange} \\\hline
  Autofocus Distance ($\mu$) & 
      \VAR{acquisitionAutoDefocusDistance} \\\hline
  Drift Meassurement  &
      \VAR{acquisitionDriftMeassurement} \\\hline % TODO: ($nm/s$)
  Delay after stage shift ($s$) & 
      \VAR{acquisitionDelayAfterStageShift} \\\hline
  Delay after image shift ($s$) &
      \VAR{acquisitionDelayAfterImageShift} \\\hline
  Maximum image shift ($\mu$) &
      \VAR{acquisitionMaxImageShift} \\\hline
  Exposures per Hole&
      \VAR{acquisitionExposureHole} \\\hline
      &\\
\end{tabulary}

\newpage
\section*{Resolution and Fit Histograms}

Workflow name = \VAR{acquisitionWorkflowName}\\
Numer of Movies = \VAR{statisticsNumberMovies}
\VAR{dataavailable}
\iflong
\begin{figure}[!htbp]
    \caption{Resolution Histogram.\label{fig:resHist}}
    \centering
    \input{\VAR{pgfResolutionFile}}
\end{figure}

\begin{figure}[!htbp]
    \caption{Defocus Histogram.\label{fig:defHist}}
    \centering
    \input{\VAR{pgfDefocusFile}}
\end{figure}
\else
No data available
\fi
\end{document}                             % The required last line
